\question

\begin{alphaparts}
    
    \questionpart
    % full rank proof
    \begin{proof}
        
        We prove the full-rank property of the Vandermonde matrix (denoted here as
        \(V_n\)) as defined via induction.
        
        Base case: \(n = 1\). We have
    
        \begin{equation}
            V_1 = \begin{bmatrix}
                1
            \end{bmatrix}
        \end{equation}
    
        is clearly of full rank. The base case is clear.
    
        Induction: Suppose \(\forall m \in \naturals\) with \(m < n\) implies
        \(V_m\) is of full rank. Then consider the matrix \(V_n\) over
        \(\{\alpha_i\}\) then for \(V_n\) to be of full rank, the vectors
        \(\vec{A_{i}} = (\alpha_i^j)\) must form a set of \(n\) linearly
        independent vectors. If possible, consider that the \(\{A_i\}\) to be
        linearly dependent instead. Then, there exist scalars \(\{c_i\}\) such
        that
        \begin{gather}
            \sum_{i = 1}^{n} c_i\vec{A_{i}} = 0 \textnormal{ , or} \\
            \forall i\; \sum_{j = 0}^{n-1} c_j \cdot \alpha_i^j = 0~.
            \label{eqn:vanderroots}
        \end{gather}
    
        Consider the polynomial formed with these \((c_i)\) as coefficients, \(f(x)
        = \sum_{j = 0}^{n-1} c_j \cdot x^j\), with \(\degree{f} = (n-1)\). We know
        of \(n\) roots of this polynomial from \autoref{eqn:vanderroots}, of the set
        \(\{\alpha_i\}\). However, this clearly violates the fact the that the
        number of roots must be less than or equal to the degree of the polynomial.
        Thus, we have a contradiction, and \(V_n\) is indeed of full rank. 
    \end{proof}

    \questionpart
    % determinant value proof
    \begin{proof}
        
        Consider the determinant of \(V_n\) as defined above instead as a
        function over one of the \(\alpha_i\), replacing it with a variable
        \(x\) in the field. As we choose \(x = \alpha_j\) for \(j < i\), we find
        that the determinant vanishes due to having two identical rows. The
        determinant being a polynomial (of degree $(n-1)$ in $x$), must have
        \((x - \alpha_j)\) as divisors. Repeating this for each \(i\), we find
        that the determinant must be divisible by the product polynomial
        \(\prod_{j < i} (\alpha_i - \alpha_j)\), i.e.
    
        \begin{equation}
            \exists q(x) \in \field[x] \;|V_n| = q(\{\alpha_k\})\cdot\prod_{j < i} (\alpha_i - \alpha_j)~.
        \end{equation}

        The product contains each \(\alpha_i\) exactly \((n-1)\) times, which
        can also by a first step expansion be verified to be the degree of the
        determinant in \(\alpha_i\). So, \(q(\alpha_k)\) may not contain any
        terms involving \(\alpha_i\). Likewise it follows for all other
        parameters. Hence, \(q(\{\alpha_k\})\) is a scalar, which by painstaking
        inductive expansion can be verified to be 1 as well.
    \end{proof}
\end{alphaparts}