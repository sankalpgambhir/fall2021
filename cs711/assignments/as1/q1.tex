\question

\begin{proof}

Given, \field~ is a field of characteristic $p$, \underline{assumed prime}. Then,
over the polynomial ring $\field[x, y]$, we evaluate the expression \((x+y)^p\).

Since field multiplication is commutative, we may write the binomial expansion
of the expression, with coefficients \(c_i\) yet undetermined as

\begin{equation}
    (x+y)^p = \sum_{i = 0}^{p} c_i \cdot x^iy^{p-i}~~.
\end{equation}

It is easy to see that \(c_i = \sum_{j = 1}^{\binom{p}{i}} 1\), \(1 \in
\field\), i.e., \(\binom{p}{i}\in\integers\) terms of the form \(x^iy^{p-i}\)
were obtained and summed. Clearly, \(c_0 = c_p = 1\), and \(\forall i \in
\naturals, \textnormal{and } i < p\), we have

\begin{equation}
    \binom{p}{i} = \frac{p!}{(p-i)!\cdot i!} = \frac{p\cdot (p-1)!}{(p-i)!\cdot i!}~~.
\end{equation}

In the factorial, and subsequently prime expansion of the denominator, every
number obtained will be smaller than \(p\), and by virtue of \(p\) being prime,
will not divide the term in the numerator, leading to \(\binom{p}{i}\) being
divisible by \(p\). And due to \(p\) being the characteristic of the field
\field, we have

\begin{gather}
    \forall i\;\exists k_i\; \binom{p}{i} = p\cdot k_i \textnormal{, and } \\
    c_i = \sum_{j = 1}^{k_i} \left( \sum_{n = 1}^{p} 1 \right) = \sum_{j = 1}^{k_i} 0 = 0~.
\end{gather}

Thus, we have \((x+y)^p = x^p + y^p\).

\end{proof}