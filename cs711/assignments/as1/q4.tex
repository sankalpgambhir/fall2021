\question

\begin{alphaparts}

    \questionpart 
        \begin{proof}
            \underline{Forward \((\Rightarrow)\) direction:}\\
            \(\alpha\) is zero of multiplicity atleast \(k\). Then \(\forall i
            \in \naturals\), with \(i \leq k\), we have 
            
            \begin{equation}
                \frac{\partial^i f}{\partial x^i}(\alpha) = 0. 
                \label{eqn:diffpart}
            \end{equation}

            In particular, for \(i = 0\) we find that \(\alpha\) is a root of
            the polynomial. And thus we can write \(f(x) = (x-\alpha)q_0(x)\)
            for some polynomial \(q_0(x)\) of degree one less than \(f(x)\).
            Differentiating this expression, we obtain the polynomial \(f'(x) =
            q_0(x) + (x - \alpha)q_0'(x)\), which due to \autoref{eqn:diffpart},
            also has \(\alpha\) as a root, and we find that in fact, \(q_0(x) =
            (x - \alpha)q_1(x)\) for a polynomial \(q_1(x)\) as before.
            Continuing this up to \(k\) differentiations we find \(\exists
            q_k(x) \in \complex[x], \textnormal{ such that } f(x) =
            (x-\alpha)^k\cdot q_k(x)\) as required per the definition of
            divisibility.

            \underline{Backward \((\Leftarrow)\) direction:}\\
            Given, \(\exists \alpha, k\) such that \((x-\alpha)^k\) divides the
            polynomial \(f(x)\), i.e. we can write for some \(q(x) \in
            \complex[x]\), \(f(x) = (x-\alpha)^k \cdot q(x)\).

            Define the set \(D_n \subseteq \complex[x]\) for each \(n\)
            containing all the polynomials from the ring which are divisible by
            \((x-\alpha)^n\). This is a descending chain of ideals. We note two
            things, first, \(\forall n, D_n \subseteq D_{n-1}\), and second,
            \(f(x) \in D_k\). I induct over \(n\) to show the desired property:

            \begin{equation}
                \forall t(x) \in D_n, \frac{\partial^n t}{\partial x^n}(\alpha) = 0~.
                \label{eqn:inductpropd}
            \end{equation}

            Also note, two corollaries of the definition, \(\forall t(x) \in
            D_n\), we have:

            \begin{enumerate}
                \item \(t'(x) \in D_{n-1}\), and
                \item \((x-\alpha)\cdot t(x) \in D_{n+1}\).
            \end{enumerate}

            Base case:
            
            \(\forall t(x) \in D_0, \textnormal{ since } t(\alpha) = 0,
            \textnormal{ then we have } \frac{\partial^0 t}{\partial x^0}(\alpha)
            = t(\alpha) = 0.\) So, the base case is clear.

            Induction:

            Given that \(\forall m \in \naturals\) with \(m < n, D_m\) satisfies
            the property in \autoref{eqn:inductpropd}, we write for some \(t(x)
            \in D_n\),

            \begin{gather}
                \exists q_t(x) \in D_{n-1} \textnormal{ such that } t(x) = (x-\alpha)q_t(x),\\ 
                \frac{\partial t}{\partial x}(x) = (x-\alpha)q_t'(x) + q_t(x),  \textnormal{ and } \\
                \frac{\partial^n t}{\partial x^n}(x) = \frac{\partial^{n-1}}{\partial x^{n-1}}(x-\alpha)q_t'(x) + \frac{\partial^{n-1}}{\partial x^{n-1}}q_t(x)~.
            \end{gather}

            Using the corollaries noted above, we see that the operands of the
            differential operators on the right in the last equation are both in
            \(D_{n-1}\), and thus the right side evaluates to zero on
            substituting \(x = \alpha\). So, 
            
            \begin{equation}
                \frac{\partial^n t}{\partial x^n}(\alpha) = 0~,
            \end{equation}

            and the induction is complete.

            Since \(f(x) \in D_k\), it satisfies the required property.

        \end{proof}

    \questionpart 
    \begin{proof}
        Due to the algebraic completeness of the complex field, and
        the Fundamental Theorem of Algebra, we can write:
    
        \begin{equation}
            f(x) = c\cdot\prod_i (x-\beta_i)^{k_i}~,
        \end{equation}
    
        where the \(\{(\beta_i, k_i)\}\) are the roots of the polynomial and
        their degrees respectively, and \(c\) is a complex scalar. It is easy to
        see from part (a) and this equation that in fact \(\mult{f}{\beta_i} =
        k_i \forall i\) and by expanding the polynomial we note \(\sum_{i} k_i =
        \degree{f}\).
    
        We can thus write
    
        \begin{gather}
            \sum_{i} \mult{f}{\beta_i} = \degree{f}~, \textnormal{ and} \label{eqn:sumroots}\\
            \mult{f}{\alpha} = 0 \textnormal{ if } \not \exists i\;\alpha = \beta_i~.
            \label{eqn:multzero}
        \end{gather}
    
        Now, for an arbitrary set \(S = \{\alpha_1, \alpha_2, ..., \alpha_n\}
        \subseteq \complex\), we write the sum of multiplicities due to
        \autoref{eqn:multzero} as
    
        \begin{equation}
            \sum_{\alpha \in S} \mult{f}{\alpha} = \sum_{\alpha \in S \cap \{\beta_i\}} \mult{f}{\alpha}~.
        \end{equation}
    
        where the right side is identified easily as a subset of the sum in
        \autoref{eqn:sumroots}, and thus we obtain
    
        \begin{equation}
            \sum_{\alpha \in S} \mult{f}{\alpha} \leq \degree{f}~.
        \end{equation}
    \end{proof}

\end{alphaparts}