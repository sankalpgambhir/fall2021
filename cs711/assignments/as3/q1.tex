\question

\begin{proof}
        Given the set \(S_{d, n}\) as defined, suppose if possible that we
        cannot find an assignment \(\mathbf{b}\) in this set for some non-zero
        polynomial \(P\) of degree \(\leq d\) in \(n\) variables such that the
        evaluation is non-vanishing, i.e., \(P(\mathbf{b}) = 0
        \forall\mathbf{b}\in S_{d, n}\).

        Write the polynomial \(P\) defined by the \(\binom{n+d}{d}\)
        coefficients \(\{a_{\alpha_1\alpha_2\ldots\alpha_n}\}\) as a linear
        functional acting on a vector of pre-evaluated monomials
        \(\{x_1^{\alpha_1}x_2^{\alpha_2}\ldots x_n^{\alpha_n}\}\) with \(\sum
        \alpha_i \leq d\),

        \begin{gather*}
            P(x_1, x_2, \ldots, x_n) = 
            \begin{pmatrix}
                a_{00\ldots 0} & a_{10\ldots 0} & \ldots & a_{00\ldots d}
            \end{pmatrix}_{1\times \binom{n+d}{d}}
            \begin{pmatrix}
                1 & x_1 & \ldots & x_n^d
            \end{pmatrix}^\top_{1\times \binom{n+d}{d}}
        \end{gather*}

        Writing the condition for \(\mathbf{b}\) in this notation, it resolves
        to the coefficient vector being acted on the right by the matrix \(M\)
        with columns as the x-vector pre-evaluated over each \(\mathbf{b}\). The
        condition assumed above is equivalent to this matrix having a
        non-trivial kernel, i.e., there existing a coefficient vector which is
        non-zero, such that all its evaluations over this set are 0.

        Consider any two distinct assignments \(A = (\alpha_1, \alpha_2, \ldots,
        \alpha_n)\) and \(B = (\beta_1, \beta_2, \ldots \beta_n)\). Suppose
        their pre-evaluated x-vectors (columns of \(M\)) are linearly dependent,
        we have

        \begin{gather*}
           \exists\;\lambda, \delta \in \complex ~\lambda \vec{X}(A) + \delta \vec{X}(B) = \vec{0}~.
        \end{gather*}

        Cherry-picking dimensions from this vector equation, first the degree 0
        condition, and finally all the degree 1 conditions, we find

        \begin{gather*}
            \lambda\cdot 1 + \delta\cdot 1 = 0~, \\ 
            \lambda = -\delta~,\\
            \vec{X}(A) = \vec{X}(B)~, \text{ and}\\
            \alpha_i = \beta_i \forall i~.
        \end{gather*}

        However, we stipulated that these assignments were distinct, so this is
        a contradiction. Thus, columns of \(M\), which represent distinct
        assignments, cannot be linearly dependent, \(M\) is of full-rank, and
        thus has a trivial kernel. This again, is a contradiction, so contrary
        to assumption, we must be able to find \(\mathbf{b} \in S_{d, n}\) for
        every non-zero polynomial \(P\) of degree \(\leq d\) in \(n\) variables
        such that it evaluates to a non-zero quantity over \(\mathbf{b}\).
    
\end{proof}

% \begin{proof}
%     Given the set \(S_{d, n}\) as defined, I induct on \(d\) and \(n\) for the
%     polynomial \(f(x_1, x_2, \ldots, x_n)\) of total degree atmost \(d\).

%     \emph{Base case:} \(n = 1, d = 1\).

%     The relevant polynomial is \(f(x_1) = a_1 x_1 + a_0\), where atleast one
%     \(a_i\) is non-zero, since \(f\) is not identically zero. If \(a_0 \not =
%     0\), set \(x_1 = 0 < d\), else set \(x_1 = 1 \leq d\). This assignment
%     results in a non-vanishing evaluation and satisfies the constraints, i.e, it
%     is in \(S_{d, n}\).

%     \emph{Induction on n:} Given a non-zero evaluating assignment can be found
%     for each degree \(c \leq d, \text{and number of variables } m < n\), I show
%     that we can find one for degree \(d\) and \(n\) variables.

%     Consider a polynomial \(f(x_1, x_2, \ldots, x_n)\). We can write, 

%     \begin{gather*}
%         f(x_1, x_2, \ldots, x_n) = \sum_{i=0}^d x_1^i f_i(x_2, x_3, \ldots, x_n)~,
%     \end{gather*}

%     for polynomials \(\{f_i\}\) in \(n-1\) variables and degrees \(\{d-i\}\)
%     respectively. Choose an assignment for the largest \(i\) for which \(f_i
%     \not \equiv 0\) such that it satisfies the constraints for \(S_{d-i, n-1}\).
%     After substitution, we obtain a univariate polynomial in \(x_1\) of degree
%     atmost \(i\). 

%     We know that we can find a non-zero evaluation for this with an assignment
%     from the set \(S_{i, 1}\). Appending this to the chosen assignment for
%     \(f_i\). This gives us a suitable assignment within \(S_{d, n}\) as the
%     first assignment constrained the sum of variables to be under \(d-i\) and
%     the appended assignment constrains to \(\leq i\), giving a total assignment
%     satisfying the condition of their sum being \(\leq d\).

%     \emph{Induction on d:} Given a non-zero evaluating assignment can be found
%     for each degree \(c < d, \text{and number of variables } m \leq n\), I show
%     that we can find one for degree \(d\) and \(n\) variables.

%     Write this time the polynomial whilst separating all degree \(d\) terms

%     \begin{gather*}
%         P(x_1, x_2, \ldots, x_n) = P_0(x_1, x_2, \ldots, x_n) + x_1P_1(x_1, x_2, \ldots, x_n) + x_2P_2(x_2, \ldots, x_n) + \ldots + x_nP_n(x_n)~,
%     \end{gather*}

%     where \(P_0\) contains all terms in \(P\) of degree \(< d\) and to construct
%     each further \(P_i\), of the remaining terms, collect those divisible by
%     \(x_i\) and collect the respective quotients, repeating with the remaining
%     indivisible terms for \(i+1\). This procedure is guaranteed to finish since
%     every term of non-zero degree is divisible by one or more of the constituent
%     variables.

%     Note that the \(\{P_i\}\) are guaranteed to be polynomials of degree \(< d\)
%     in \(\leq n\) variables. Find the smallest \(i\) for which \(P_i \not \equiv
%     0\). This is possible since \(P\) is non-zero. Fix first \(x_i = 1\). Now,
%     consider an assignment for this polynomial \(P_i(1, x_{i+1}, \ldots, x_n)\)
%     satisfying the conditions for \(S_{d-1, n-i+1}\). Append \(x_i = 1\) and
%     \(x_j = 0 \forall j < i\) to this assignment. This generates a non-zero
%     assignment for \(P\) in \(S_{d, n}\).

%     Note: there is an edge case here, where \(i = n\). i.e. \(P(\mathbf{x}) =
%     x_n P_n(x_n)\). This procedure sets \(x_n = 1\), but we could have \(P_n(1)
%     = 0\) as well. However, this scenario produces \(P \equiv 0\). 

% \end{proof}
