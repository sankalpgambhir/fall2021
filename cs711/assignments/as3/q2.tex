\question

\newcommand{\bvar}{\mathbf{b}}

\begin{proof}
    Given the set \(T_{d, n}\) as defined, suppose if possible, that we cannot
    find any vector in the set such that a given non-zero polynomial \(P\) of
    degree \(d\) in \(n\) variables evaluates to a non-zero value on it. I prove
    that this leads to a contradiction.

    The evaluation condition implies \(P(\bvar) = 0\) for every \(\bvar \in
    T_{d, n}\). Instead of writing it as a polynomial system, evaluate each
    possible variable in \(P(x_1, x_2, \ldots, x_n)\) and write it as the
    following linear system

    \begin{gather*}
        \begin{pmatrix}
            a_{00\ldots 0} & a_{10\ldots 0} & \ldots & a_{00\ldots 0 d}
        \end{pmatrix}
        \begin{pmatrix}
            x_1^0x_2^0\ldots x_n^0 & x_1^1x_2^0\ldots x_n^0 & \ldots & x_1^0x_2^0\ldots x_n^d
        \end{pmatrix}^\top
    \end{gather*}

    This linear system has dimension \(\binom{n+d}{d}\). Given that we have
    evaluations on \(\binom{n+d}{d}\) points as well, we can write this as the
    linear system

    \begin{gather*}
        \begin{pmatrix}
            a_{00\ldots 0} & a_{10\ldots 0} & \ldots & a_{00\ldots 0 d}
        \end{pmatrix}
        \begin{pmatrix}
            p_1^{0\cdot 0}p_2^{0 \cdot 0}\ldots p_n^{0 \cdot 0} & p_1^{0\cdot 1}p_2^{0 \cdot 1}\ldots p_n^{0 \cdot 1} & \ldots & p_1^{0\cdot t}p_2^{0 \cdot t}\ldots p_n^{0 \cdot t}\\
            p_1^{1\cdot 0}p_2^{0 \cdot 0}\ldots p_n^{0 \cdot 0} & p_1^{1\cdot 1}p_2^{0 \cdot 1}\ldots p_n^{0 \cdot 1} & \ldots & p_1^{1\cdot t}p_2^{0 \cdot t}\ldots p_n^{0 \cdot t}\\
            \vdots & \ddots & \ddots & \vdots\\
            p_1^{0\cdot 0}p_2^{0 \cdot 0}\ldots p_n^{d \cdot 0} & p_1^{0\cdot 1}p_2^{0 \cdot 1}\ldots p_n^{d \cdot 1} & \ldots & p_1^{0\cdot t}p_2^{0 \cdot t}\ldots p_n^{d \cdot t}\\
        \end{pmatrix} = 
        \begin{pmatrix}
            0 & 0 \ldots & 0
        \end{pmatrix}~,
    \end{gather*}

    with \(t = \binom{n+d}{d}\). This matrix must have a non-trvial kernel,
    since the coefficients were constrained to be non-zero. However, note that
    this is the Vandermonde matrix in the \(\binom{n+d}{d}\) integers
    \(\{p_1^{\alpha_1}p_2^{\alpha_2}\ldots p_n^{\alpha_n}\}\), with \(\sum
    \alpha_i \leq d\). Since these are distinct primes, there are no duplicates
    in this set, and thus the determinant of the Vandermonde matrix must be
    non-zero. However, this would mean that it is of full rank and has a trivial
    kernel, this is a contradiction. 
    
    Thus, our assumption of being unable to produce a non-zero
    evaluation over the given set was incorrect, and there exists a vector in
    \(T_{d, n}\) for every polynomial of maximum total degree \(d\) and \(n\)
    variables such that it evaluates to a non-zero complex quantity over the
    given vector provided that it is not identically zero.

\end{proof}
