\begin{frame}
    \frametitle{Topological Spaces}
    The following questions deal with the idea of Topological Spaces, so here's a quick recap on what exactly those are.\\\\
    \textbf{Topological Spaces:} A \textit{topological space} is a set $X$ on which a \textit{topology} $\tau$ is equipped. $\tau$ is a collection of subsets of $X$ (or, $\tau$ is a subset of the power set $2^{X}$ of $X$) such that -

    \begin{enumerate}
        \item $\varnothing$ and $X$ should belong to $\tau$
        \item the union of the elements in any subset of $\tau$ should belong to $\tau$
        \item the intersection of the elements in any finite subset of $\tau$ should belong to $\tau$
    \end{enumerate}
\end{frame}

\begin{frame}
    The elements of $\tau$ are called \textit{open sets}. Thus, a topological space is a pair $(X, \tau)$ consisting of a set and a topology on it.\\\\
    We can reframe the axioms given on the previous slide in terms of open sets - 
    \begin{enumerate}
        \item The empty and the full set are open.
        \item Any arbitrary union of open sets is open.
        \item Any finite intersection of open sets is open.
    \end{enumerate}
\end{frame}

\begin{frame}
    \frametitle{Problem Set 3.4 - 8}
    \textbf{Show that: The euclidean, diamond, square metrics on $\mathbb{R}^2$ have the same underlying topology. (When we say continuous map from $\mathbb{R}^2$ to $\mathbb{R}$, it is w.r.t this topology.) Further, check that it coincides with the product topology on $\mathbb{R} \times \mathbb{R}$.}\\\\
    \pause
    Before we jump into the proof for this, we need to talk about how to compare topologies. The set of all topologies on a set forms a partially ordered set with the binary relation $\subseteq$. With this relation, we can define a partial ordering that we use to compare topologies. \\\\
    \pause
    If there are two topologies $\tau$ and $\tau'$ on $X$ such that $\tau \subseteq \tau'$, then $\tau$ is said to be a \textit{coarser or weaker topology} than $\tau'$ and $\tau'$ is a \textit{finer or stronger topology} than $\tau$. An additional check is whether the two topologies are equal, if they aren't equal, then one can be called \textbf{strictly} finer or coarser than the other.
\end{frame}

\begin{frame}
    \frametitle{Lemma 13.3 from Munkres' Toplogy}
    Let $\mathcal{B} \text{ and } \mathcal{B}'$ be the bases for the topologies $\tau \text{ and } \tau'$ on $X$. Then the following are equivalent -
    \begin{enumerate}
        \item $\tau'$ is finer than $\tau$
        \item for each $x \in X$ and each basis element $B \in \mathcal{B}$ containing $x$, there is a basis element $B' \in \mathcal{B}'$ such that $x \in B' \subset B$
    \end{enumerate}
\end{frame}

\begin{frame}
    \frametitle{The Metrics}
    The three metrics in question were defined in the earlier slides, but for the sake of context, the diamond metric $d_1$, the euclidean metric $d_2$ and the square metric $d_\infty$ are defined over $\mathbb{R}^2$ as follows - (note that the notation used is $\boldsymbol{x} = (x_1, x_2)$, $\boldsymbol{y} = (y_1, y_2)$ are points in $\mathbb{R}^2$)
    \begin{gather*}
        d_1(\boldsymbol{x}, \boldsymbol{y}) = |x_1 - y_1| + |x_2 - y_2| \\
        d_2(\boldsymbol{x}, \boldsymbol{y}) = \sqrt{|x_1 - y_1|^2 + |x_2 - y_2|^2} \\
        d_\infty(\boldsymbol{x}, \boldsymbol{y}) = \text{max} \{|x_1 - y_1|, |x_2 - y_2|\}
    \end{gather*}
    Let $\tau_{\text{square}}$, $\tau_{\text{euclidean}}$ and $\tau_{\text{diamond}}$ be the respective topologies generated by the square, euclidean and diamond metrics.
\end{frame}

\begin{frame}
    \frametitle{Problem Set 3.4 - 8}
    Now, let us examine the relation between these three metrics.
    \begin{equation*}
        d_\infty(\boldsymbol{x}, \boldsymbol{y}) = \text{max} \{|x_1 - y_1|, |x_2 - y_2|\} = \left(\text{max} \{|x_1 - y_1|, |x_2 - y_2|\}^2 \right)^{\frac{1}{2}}
    \end{equation*}
    \pause
    We can also straightaway see that -
    \begin{equation*}
        \text{max} \{|x_1 - y_1|, |x_2 - y_2|\}^2 \leq |x_1 - y_1|^2 + |x_2 - y_2|^2
    \end{equation*}
    \pause
    Note the fact that $f(x) = x^{\frac{1}{2}}$ is an increasing function for $x \geq 0$. Thus, we have -
    \begin{gather*}
        \left(\text{max} \{|x_1 - y_1|, |x_2 - y_2|\}^2 \right)^{\frac{1}{2}} \leq \sqrt{|x_1 - y_1|^2 + |x_2 - y_2|^2} \\
        \implies d_\infty(\boldsymbol{x}, \boldsymbol{y}) \leq d_2(\boldsymbol{x}, \boldsymbol{y})
    \end{gather*} 
\end{frame}

\begin{frame}
    Now,
    \begin{gather*}
        d_1(\boldsymbol{x}, \boldsymbol{y}) = |x_1 - y_1| + |x_2 - y_2| = \sqrt{\left(|x_1 - y_1| + |x_2 - y_2|\right)^2} \\
        = \sqrt{\left(|x_1 - y_1|^2 + |x_2 - y_2|^2 + 2\cdot|x_1 - y_1|\cdot|x_2 - y_2|\right)} \geq \sqrt{|x_1 - y_1|^2 + |x_2 - y_2|^2} \\
        \implies d_1(\boldsymbol{x}, \boldsymbol{y}) \geq d_2(\boldsymbol{x}, \boldsymbol{y})
    \end{gather*}
    (since \( f(x) = \sqrt{x} \) is an increasing function, and \( 2\cdot|x_1 - y_1|\cdot|x_2 - y_2| \geq 0 \) \\\\
    \pause
    And since \(d_\infty(\boldsymbol{x}, \boldsymbol{y}) \leq d_2(\boldsymbol{x}, \boldsymbol{y})\), we have the following ordering -
    \begin{equation*}
        d_\infty(\boldsymbol{x}, \boldsymbol{y}) \leq d_2(\boldsymbol{x}, \boldsymbol{y}) \leq d_1(\boldsymbol{x}, \boldsymbol{y})
    \end{equation*}
\end{frame}

\begin{frame}
    Now, consider the following - 
    \begin{gather*}
        d_\infty(\boldsymbol{x}, \boldsymbol{y}) = \text{max} \{|x_1 - y_1|, |x_2 - y_2|\} \geq |x_1 - y_1| \\
        \text{Also, } \text{max} \{|x_1 - y_1|, |x_2 - y_2|\} \geq |x_2 - y_2| \\
        \implies 2 \cdot\text{max} \{|x_1 - y_1|, |x_2 - y_2|\} \geq |x_1 - y_1| + |x_2 - y_2| \\
        \implies 2 \cdot d_\infty(\boldsymbol{x}, \boldsymbol{y}) \geq d_1(\boldsymbol{x}, \boldsymbol{y})
    \end{gather*}
    Which implies, we have the following ordering -
    \begin{equation*}
        d_\infty(\boldsymbol{x}, \boldsymbol{y}) \leq d_2(\boldsymbol{x}, \boldsymbol{y}) \leq d_1(\boldsymbol{x}, \boldsymbol{y}) \leq 2 \cdot d_\infty(\boldsymbol{x}, \boldsymbol{y})
    \end{equation*}
\end{frame}

\begin{frame}
    Now, we have 4 metrics \( d_\infty(\boldsymbol{x}, \boldsymbol{y}), d_2(\boldsymbol{x}, \boldsymbol{y}), d_1(\boldsymbol{x}, \boldsymbol{y}) \text{ and } 2*d_\infty(\boldsymbol{x}, \boldsymbol{y})\). (it is a trivial check to see that $2 \cdot d_\infty(\boldsymbol{x}, \boldsymbol{y})$ also forms a metric). Before looking at the notion of open sets mathematically, we take a visual look at what open balls of the same radius look like, w.r.t each of these metrics. \\
    \pause
    \begin{figure}
        \hspace{8em}
        \scalebox{1.2}{\input{fig/parth/balls.pdf_tex}}
    \end{figure}
\end{frame}

\begin{frame}
    Now we look at this mathematically. Consider the following ordering -
    \begin{equation*}
        d_\infty(\boldsymbol{x}, \boldsymbol{y}) \leq d_2(\boldsymbol{x}, \boldsymbol{y}) \leq 2*d_\infty(\boldsymbol{x}, \boldsymbol{y})
    \end{equation*}
    \pause
    For all $\boldsymbol{x} \in \mathbb{R}^2$ and $\epsilon > 0$, since $d_2(\boldsymbol{x}, \boldsymbol{y}) \leq 2 \cdot d_\infty(\boldsymbol{x}, \boldsymbol{y})$,
    \begin{gather*}
         2 \cdot d_\infty(\boldsymbol{x}, \boldsymbol{y}) < \epsilon \implies d_2(\boldsymbol{x}, \boldsymbol{y}) < \epsilon \\
         \text{Thus, }d_\infty(\boldsymbol{x}, \boldsymbol{y}) < \frac{\epsilon}{2} \implies d_2(\boldsymbol{x}, \boldsymbol{y}) < \epsilon
    \end{gather*}
    \pause
    This means that the open ball $B_{d_\infty}\left(\boldsymbol{x}, \frac{\epsilon}{2}\right)$ in the topology induced by $d_{\infty}$ is contained within the open ball $B_{d_2}\left(\boldsymbol{x}, \epsilon\right)$ in the topology induced by $d_2$. Implying that the square metric topology is \textit{finer} than the Euclidean metric topology. \\\\
    (since you've shown that every possible open set in the topology induced by $d_2$ will be present in the topology induced by $d_\infty \implies \tau_{\text{euclidean}} \subseteq \tau_{\text{square}}$)
\end{frame}

\begin{frame}
    Similarly, since $d_\infty(\boldsymbol{x}, \boldsymbol{y}) \leq d_2(\boldsymbol{x}, \boldsymbol{y})$, we have -
    \begin{equation*}
        d_2(\boldsymbol{x}, \boldsymbol{y}) < \epsilon \implies d_\infty(\boldsymbol{x}, \boldsymbol{y}) < \epsilon
    \end{equation*}
    \pause
    This means that the open ball $B_{d_2}\left(\boldsymbol{x}, \epsilon\right)$ in the topology induced by $d_2$ is contained within the open ball $B_{d_\infty}\left(\boldsymbol{x}, \epsilon\right)$ in the topology induced by $d_\infty$. Implying that the Euclidean metric topology is \textit{finer} than the square metric topology. \\\\
    (since you've shown that every possible open set in the topology induced by $d_\infty$ will be present in the topology induced by $d_2 \implies \tau_{\text{square}} \subseteq \tau_{\text{euclidean}}$) \\\\
    \pause
    Combining the two results together, we have shown that 
    \begin{equation*}
        \tau_{\text{square}} \subseteq \tau_{\text{euclidean}} \subseteq \tau_{\text{square}} \implies \tau_{\text{square}} = \tau_{\text{euclidean}}
    \end{equation*}
\end{frame}

\begin{frame}
    We perform the same process as earlier. We have the following ordering -
    \begin{equation*}
        d_\infty(\boldsymbol{x}, \boldsymbol{y}) \leq d_1(\boldsymbol{x}, \boldsymbol{y}) \leq 2*d_\infty(\boldsymbol{x}, \boldsymbol{y})
    \end{equation*}
    \pause
    For all $\boldsymbol{x} \in \mathbb{R}^2$ and $\epsilon > 0$, since $d_1(\boldsymbol{x}, \boldsymbol{y}) \leq 2*d_\infty(\boldsymbol{x}, \boldsymbol{y})$,
    \begin{gather*}
         2*d_\infty(\boldsymbol{x}, \boldsymbol{y}) < \epsilon \implies d_1(\boldsymbol{x}, \boldsymbol{y}) < \epsilon \\
         \text{Thus, }d_\infty(\boldsymbol{x}, \boldsymbol{y}) < \frac{\epsilon}{2} \implies d_1(\boldsymbol{x}, \boldsymbol{y}) < \epsilon
    \end{gather*}
    \pause
    This means that the open ball $B_{d_\infty}\left(\boldsymbol{x}, \frac{\epsilon}{2}\right)$ in the topology induced by $d_{\infty}$ is contained within the open ball $B_{d_1}\left(\boldsymbol{x}, \epsilon\right)$ in the topology induced by $d_1$. Implying that the square metric topology is \textit{finer} than the diamond metric topology. \\\\
    (since you've shown that every possible open set in the topology induced by $d_1$ will be present in the topology induced by $d_\infty \implies \tau_{\text{diamond}} \subseteq \tau_{\text{square}}$)
\end{frame}

\begin{frame}
    Similarly, since $d_\infty(\boldsymbol{x}, \boldsymbol{y}) \leq d_1(\boldsymbol{x}, \boldsymbol{y})$, we have -
    \begin{equation*}
        d_1(\boldsymbol{x}, \boldsymbol{y}) < \epsilon \implies d_\infty(\boldsymbol{x}, \boldsymbol{y}) < \epsilon
    \end{equation*}
    \pause
    This means that the open ball $B_{d_1}\left(\boldsymbol{x}, \epsilon\right)$ in the topology induced by $d_1$ is contained within the open ball $B_{d_\infty}\left(\boldsymbol{x}, \epsilon\right)$ in the topology induced by $d_\infty$. Implying that the diamond metric topology is \textit{finer} than the square metric topology. \\\\
    (since you've shown that every possible open set in the topology induced by $d_\infty$ will be present in the topology induced by $d_1 \implies \tau_{\text{square}} \subseteq \tau_{\text{diamond}}$) \\\\
    \pause
    Combining the two results together, we have shown that 
    \begin{equation*}
        \tau_{\text{square}} \subseteq \tau_{\text{diamond}} \subseteq \tau_{\text{square}} \implies \tau_{\text{square}} = \tau_{\text{diamond}}
    \end{equation*}
\end{frame}

\begin{frame}
    Thus, we have shown that $\tau_{\text{square}} = \tau_{\text{diamond}} \text{ and } \tau_{\text{square}} = \tau_{\text{euclidean}} \implies \tau_{\text{square}} = \tau_{\text{euclidean}} = \tau_{\text{diamond}} \, \qedsymbol$ \\\\
    \pause
    We now need to show that they coincide with the product topology on $\mathbb{R} \times \mathbb{R}$. We show this for the square metric as follows. \\\\
    \pause
    Let $\boldsymbol{B} = (a_1,b_1) \times (a_2,b_2)$ be a basis element for the product topology on $\mathbb{R} \times \mathbb{R}$, let $\boldsymbol{x} = (x_1, x_2)$ be some element of $\boldsymbol{B}$. Now, by the definition of open sets, there exist $\epsilon_1 \text{ and } \epsilon_2$ such that -
    \begin{equation*}
        (x_1 - \epsilon_1,x_1 + \epsilon_1) \subset (a_1,b_1) \text{ and }(x_2 - \epsilon_2,x_2 + \epsilon_2) \subset (a_2,b_2)
    \end{equation*}
    \pause
    Choose $\epsilon = \text{min} \{\epsilon_1, \epsilon_2\}$. Then $B_{d_\infty}(\boldsymbol{x},\epsilon) \subset \boldsymbol{B}$ (since $B_{d_\infty}(\boldsymbol{x},\epsilon) = (x_1 - \epsilon,x_1 + \epsilon) \times (x_2 - \epsilon,x_2 + \epsilon)\subset (a_1,b_1) \times (a_2,b_2)$).\\
    \pause
    Hence, we have shown that each open set in the product topology will be present within the square metric topology. Hence, $\tau_{\text{product}} \subseteq \tau_{\text{square}}$
\end{frame}

\begin{frame}
    Now, let $B_{d_{\infty}}(\boldsymbol{x},\epsilon)$ be an arbitrary open ball in $\mathbb{R}^2$ with the square metric topology $\tau_{\text{square}}$. This open ball is a basis element for the square metric topology (since the basis elements of metric-induced topologies are the open balls generated by that metric). But -
    \begin{equation*}
        B_{d_{\infty}}(\boldsymbol{x},\epsilon) = (x_1 - \epsilon, x_1 + \epsilon) \times (x_2 - \epsilon, x_2 + \epsilon)
    \end{equation*}
    \pause
    is a basis element for the product topology as well! (since the basis elements of the product topology are arbitrary cartesian products of open sets in $\mathbb{R}$). This implies that every open set in the topology induced by the square metric will be present in the product topology. Hence $\tau_{\text{square}} \subseteq \tau_{\text{product}}$.\\\\
    \pause
    Combining the two results, $\tau_{\text{square}} \subseteq \tau_{\text{product}} \subseteq \tau_{\text{square}} \implies \tau_{\text{square}} = \tau_{\text{product}}$.
    \begin{equation*}
        \tau_{\text{square}} = \tau_{\text{euclidean}} = \tau_{\text{diamond}} = \tau_{\text{product}} \, \qedsymbol
    \end{equation*}
\end{frame}