\begin{frame}
    \frametitle{Problem 3.6}

    \textbf{Problem. } Suppose \(E\) is a vector bundle over \(M\), and \(U\) is
    an open set containing \(p \in M\). Let \(s\) be a smooth section of
    \(E_U\). Show that there exists a smooth section \(t\) of \(E\) whose
    restriction to \(U\) agrees with \(s\) in a neighbourhood of \(p\).

\end{frame}

% define vector bundle and sections?

\begin{frame}
    \frametitle{Smooth Sections}

    A smooth section of a vector bundle \(\pi: E \to M\) is a smooth map \(s: M
    \to E\) such that \(\pi \circ s = \textsf{id}\). \\

    The space of sections over a vector bundle \(E\), \(\Gamma(E)\) carries the
    structure of a vector space.

\end{frame}

\begin{frame}
    
    Given a smooth section \(s\) of \(E_U\), construct a smooth function \(f\)
    over \(M\) such that it is uniformly \(1\) in a neighbourhood of \(p\) and
    \(0\) everywhere outside \(U\).  \pause

    Note that \(f\cdot s\) is a smooth section of \(E_U\) as well. As it
    vanishes along the boundary, we can trivially extend it to a global smooth
    section \pause

    \begin{gather}
        t(x) = \begin{cases}
            f(x)\cdot s(x) \text{ if } x \in U \\
            0 \text{ otherwise.}
        \end{cases}
    \end{gather}

\end{frame}

\begin{frame}
    \frametitle{Constructing the smooth function}

    Consider the following construction, beginning with the standard mollifier, 
    
    \begin{gather*}
        g(x) = e^{\frac{1}{1-x^2}}~, \\
        f(x) = \begin{cases}
            1 \text{ if } x \leq 0 \\
            e^{1 - \frac{1}{1-x^2}} \text{ if } 0 < x < 1 \\
            0 \text{ if } x \geq 1 \\
        \end{cases}~.
    \end{gather*} \pause

    Note that \(f\) is indeed smooth. By transforming and taking products of
    functions of this kind, we can construct a required smooth function in
    \(\reals^n\) with \(n\) being the dimension of our manifold.

\end{frame}

\begin{frame}

    Example choice in \(\reals\):
    
    \begin{center}
        \begin{tikzpicture}
 
            \begin{axis}[
                xmin = -1, xmax = 2,
                ymin = -0.5, ymax = 1.5]
                \addplot[
                    domain = 0:1,
                    samples = 10,
                    smooth,
                    thick,
                    blue,
                ] {exp((1/(x^2-1))+1)};
                \addplot[
                    domain = -1:0,
                    samples = 5,
                    smooth,
                    thick,
                    blue,
                ] {1};
                \addplot[
                    domain = 1:2,
                    samples = 5,
                    smooth,
                    thick,
                    blue,
                ] {0};
            \end{axis}
             
            \end{tikzpicture}
    \end{center}

\end{frame}

\begin{frame}

    For an interval
    
    \begin{center}
        \begin{tikzpicture}
 
            \begin{axis}[
                xmin = -3, xmax = 2,
                ymin = -0.5, ymax = 1.5]
                \addplot[
                    domain = 0:1,
                    samples = 10,
                    smooth,
                    thick,
                    blue,
                ] {exp((1/(x^2-1))+1)};
                \addplot[
                    domain = -1:0,
                    samples = 5,
                    smooth,
                    thick,
                    blue,
                ] {1};
                \addplot[
                    domain = 1:2,
                    samples = 5,
                    smooth,
                    thick,
                    blue,
                ] {0};
                \addplot[
                    domain = -3:-1.97,
                    samples = 5,
                    smooth,
                    thick,
                    blue,
                ] {0};
                \addplot[
                    domain = -2:-1,
                    samples = 40,
                    smooth,
                    thick,
                    blue,
                ] {exp(((1/((-x-1)^2-1))+1))};
            \end{axis}
             
            \end{tikzpicture}
    \end{center}

\end{frame}

\begin{frame}
    
    By construction, it agrees with \(s\) in a neighbourhood of \(p\) when
    restricted. \(t\) is the required section. \pause

    This only matches on a neighbourhood, can we do better? Can we define a
    section on a slice of the space and extend it globally?

\end{frame}

\begin{frame}
    \frametitle{Global Sections (lack thereof)}

    Consider a trivial bundle on \(\reals\) given by \(\reals \times \reals\).
    Construct a section on \(\reals \setminus \{0\}\) given by \(s(p) = (p,
    \frac{1}{p})\).\\

    Can we extend this to a section on the whole space? \\ \pause

    Note that we can still construct a global section that agrees with this
    section in a neighbourhood of any interior point of \(\reals \setminus
    \{0\}\).

\end{frame}